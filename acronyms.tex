\addtocounter{table}{-1}
\begin{longtable}{|p{0.145\textwidth}|p{0.8\textwidth}|}\hline
\textbf{Acronym} & \textbf{Description}  \\\hline

DM & \gls{Data Management} \\\hline
DMLT & \gls{DM} Leadership Team \\\hline
Data Management & The \gls{LSST} \gls{Subsystem} responsible for the \gls{Data Management} System (\gls{DMS}), which will capture, store, catalog, and serve the \gls{LSST} dataset to the scientific community and public. The DM team is responsible for the \gls{DMS} architecture, applications, middleware, infrastructure, algorithms, and Observatory Network Design. DM is a distributed team working at \gls{LSST} and partner institutions, with the DM \gls{Subsystem} Manager located at \gls{LSST} headquarters in Tucson \\\hline
Document & Any object (in any application supported by \gls{DocuShare} or design archives such as PDMWorks or GIT) that supports project management or records milestones and deliverables of the \gls{LSST} Project \\\hline
EPO & \gls{Education and Public Outreach} \\\hline
Education and Public Outreach & The \gls{LSST} subsystem responsible for the cyberinfrastructure, user interfaces, and outreach programs necessary to connect educators, planetaria, citizen scientists, amateur astronomers, and the general public to the transformative \gls{LSST} dataset \\\hline
FITS & \gls{Flexible Image Transport System} \\\hline
HSC & Hyper Suprime-Cam \\\hline
Handle & The unique identifier assigned to a document uploaded to \gls{DocuShare} \\\hline
LDM & \gls{LSST} \gls{Data Management} (\gls{Document} \gls{Handle}) \\\hline
LSST & Large Synoptic Survey Telescope \\\hline
Scope & The work needed to be accomplished in order to deliver the product, service, or result with the specified features and functions \\\hline
WG & Working Group \\\hline
calibration & The process of translating signals produced by a measuring instrument such as a telescope and \gls{camera} into physical units such as \gls{flux}, which are used for scientific analysis. Calibration removes most of the contributions to the signal from environmental and instrumental factors, such that only the astronomical component remains \\\hline
camera & An imaging device mounted at a telescope focal plane, composed of optics, a shutter, a set of filters, and one or more sensors arranged in a focal plane array \\\hline
flux & Shorthand for radiative \gls{flux}, it is a measure of the transport of radiant energy per unit area per unit time. In astronomy this is usually expressed in cgs units: erg/cm2/s \\\hline
metadata & General term for data about data, e.g., attributes of astronomical objects (e.g. images, sources, astroObjects, etc.) that are characteristics of the objects themselves, and facilitate the organization, preservation, and query of data sets. (E.g., a \gls{FITS} header contains \gls{metadata}) \\\hline
provenance & Information about how \gls{LSST} images, Sources, and Objects were created (e.g., versions of pipelines, algorithmic components, or templates) and how to recreate them \\\hline
\end{longtable}
