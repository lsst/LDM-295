\addtocounter{table}{-1}
\begin{longtable}{|p{0.145\textwidth}|p{0.8\textwidth}|}\hline
\textbf{Acronym} & \textbf{Description}  \\\hline

DM & \gls{Data Management} \\\hline
DMLT & \gls{DM} Leadership Team \\\hline
EPO & \gls{Education and Public Outreach} \\\hline
HSC & Hyper Suprime-Cam \\\hline
LDM & \gls{LSST} \gls{Data Management} (\gls{Document} \gls{Handle}) \\\hline
Scope & The work needed to be accomplished in order to deliver the product, service, or result with the specified features and functions \\\hline
WG & Working Group \\\hline
calibration & The process of translating signals produced by a measuring instrument such as a telescope and \gls{camera} into physical units such as \gls{flux}, which are used for scientific analysis. Calibration removes most of the contributions to the signal from environmental and instrumental factors, such that only the astronomical component remains \\\hline
metadata & General term for data about data, e.g., attributes of astronomical objects (e.g. images, sources, astroObjects, etc.) that are characteristics of the objects themselves, and facilitate the organization, preservation, and query of data sets. (E.g., a \gls{FITS} header contains \gls{metadata}) \\\hline
provenance & Information about how \gls{LSST} images, Sources, and Objects were created (e.g., versions of pipelines, algorithmic components, or templates) and how to recreate them \\\hline
\end{longtable}
